%%% Intro.tex --- 
%% 
%% Filename: Intro.tex
%% Description: 
%% Author: Ola Leifler
%% Maintainer: 
%% Created: Thu Oct 14 12:54:47 2010 (CEST)
%% Version: $Id$
%% Version: 
%% Last-Updated: Thu May 19 14:12:31 2016 (+0200)
%%           By: Ola Leifler
%%     Update #: 5
%% URL: 
%% Keywords: 
%% Compatibility: 
%% 
%%%%%%%%%%%%%%%%%%%%%%%%%%%%%%%%%%%%%%%%%%%%%%%%%%%%%%%%%%%%%%%%%%%%%%
%% 
%%% Commentary: 
%% 
%% 
%% 
%%%%%%%%%%%%%%%%%%%%%%%%%%%%%%%%%%%%%%%%%%%%%%%%%%%%%%%%%%%%%%%%%%%%%%
%% 
%%% Change log:
%% 
%% 
%% RCS $Log$
%%%%%%%%%%%%%%%%%%%%%%%%%%%%%%%%%%%%%%%%%%%%%%%%%%%%%%%%%%%%%%%%%%%%%%
%% 
%%% Code:


\chapter{Introduktion}
\label{cha:introduction}
\section{Motivation}
\label{sec:motivation}
Att komprimera filer är väldigt användbart i dagens samhälle. Dels för att spara filer i ett mindre format, vilket leder till de tar upp mindre plats på lagringsenheten, och för att det går snabbare att överföra informationen mellan olika användare. Det finns flertalet metoder för att komprimera filer och olika metoder passar bättre till vissa filtyper. Den här rapporten kommer att undersöka två olika komprimeringsmetoder, Huffman-kodning och Lempel-Ziv-Welch-kodning (LZW).

\section{Mål}
\label{sec:aim}

Målet med denna rapport är att undersöka två olika komprimeringsmetoder, Huffman-kodning och LZW-kodning. Hur fungerar dessa komprimeringsmetoder och vilka olika fördelar och nackdelar finns, samt hur bra de presterar vid komprimering av olika filtyper.

\section{Frågeställningar}
\label{sec:research-questions}

De frågeställningar som denna rapport kommer att behandla listas nedan.

\begin{enumerate}
\item Hur stor plats på lagringsenheten sparar vi då en fil komprimeras med Huffman-kodning respektive LZW-kodning.

\item Hur nära kommer vi den optimala takten (entropin) när vi komprimerar en fil med Huffman-kodning respektive LZW-kodning.

\item Vilka filtyper passar bäst att komprimera med Huffman-kodning respektive LZW-kodning.

\end{enumerate}


\section{Avgränsningar}
\label{sec:delimitations}

Det finns flera olika metoder för komprimering av filer. Denna rapport undersöker endast Huffman-kodning och LZW-kodning och tar ej hänsyns till övriga existerande metoder för komprimering. Tilläggas bör att vid referering till Huffman-kodning menas endast statisk Huffman-kodning, det vill säga då källans frekvenser är kända på förhand och uppskattas inte samtidigt som filen kodas.

%\nocite{scigen}
%We have included Paper \ref{art:scigen}

%%%%%%%%%%%%%%%%%%%%%%%%%%%%%%%%%%%%%%%%%%%%%%%%%%%%%%%%%%%%%%%%%%%%%%
%%% Intro.tex ends here


%%% Local Variables: 
%%% mode: latex
%%% TeX-master: "demothesis"
%%% End: 
