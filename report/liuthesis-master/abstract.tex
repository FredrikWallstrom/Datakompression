%%% Abstract.tex --- 
%% 
%% Filename: Abstract.tex
%% Description: 
%% Author: Ola Leifler
%% Maintainer: 
%% Created: Thu Oct 14 13:34:11 2010 (CEST)
%% Version: $Id$
%% Version: 
%% Last-Updated: Tue Dec  1 15:19:52 2015 (+0100)
%%           By: Ola Leifler
%%     Update #: 4
%% URL: 
%% Keywords: 
%% Compatibility: 
%% 
%%%%%%%%%%%%%%%%%%%%%%%%%%%%%%%%%%%%%%%%%%%%%%%%%%%%%%%%%%%%%%%%%%%%%%
%% 
%%% Commentary: 
%% 
%% 
%% 
%%%%%%%%%%%%%%%%%%%%%%%%%%%%%%%%%%%%%%%%%%%%%%%%%%%%%%%%%%%%%%%%%%%%%%
%% 
%%% Change log:
%% 
%% 
%% RCS $Log$
%%%%%%%%%%%%%%%%%%%%%%%%%%%%%%%%%%%%%%%%%%%%%%%%%%%%%%%%%%%%%%%%%%%%%%
%% 
%%% Code:

\noindent Entropi-estimering, Huffman-kodning och LZW-kodning implementerades i ett C++ program i kursen TSBK08 - Datakompression. Att komprimera filer i dagens samhälle är viktigt för att spara plats på lagringsenheten, det känns således viktigt att lära sig hur komprimering med vissa metoder fungerar och hur bra de presterar. Huffman-kodning ger en optimal komprimerad fil och passar bra att komprimera de flesta filtyperna med. LZW-kodning är en komprimeringsmetod som passar bra för att komprimera stora filer med symbolsekvenser som upprepas ofta. Huffman-kodning har sina nackdelar jämfört med LZW-kodning, vi behöver spara statistik över hur källan ser ut. LZW-kodning har också sina nackdelar jämfört med Huffman-kodning, den ger inte en optimal kod utan behöver ha en stor källa med många upprepade symboler för att bli effektiv.

%%%%%%%%%%%%%%%%%%%%%%%%%%%%%%%%%%%%%%%%%%%%%%%%%%%%%%%%%%%%%%%%%%%%%%
%%% Abstract.tex ends here


%%% Local Variables: 
%%% mode: latex
%%% TeX-master: "demothesis"
%%% End: 
