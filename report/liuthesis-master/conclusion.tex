%%% lorem.tex --- 
%% 
%% Filename: lorem.tex
%% Description: 
%% Author: Ola Leifler
%% Maintainer: 
%% Created: Wed Nov 10 09:59:23 2010 (CET)
%% Version: $Id$
%% Version: 
%% Last-Updated: Wed Nov 10 09:59:47 2010 (CET)
%%           By: Ola Leifler
%%     Update #: 2
%% URL: 
%% Keywords: 
%% Compatibility: 
%% 
%%%%%%%%%%%%%%%%%%%%%%%%%%%%%%%%%%%%%%%%%%%%%%%%%%%%%%%%%%%%%%%%%%%%%%
%% 
%%% Commentary: 
%% 
%% 
%% 
%%%%%%%%%%%%%%%%%%%%%%%%%%%%%%%%%%%%%%%%%%%%%%%%%%%%%%%%%%%%%%%%%%%%%%
%% 
%%% Change log:
%% 
%% 
%% RCS $Log$
%%%%%%%%%%%%%%%%%%%%%%%%%%%%%%%%%%%%%%%%%%%%%%%%%%%%%%%%%%%%%%%%%%%%%%
%% 
%%% Code:

\chapter{Slutsats}
\label{cha:conclusion}

Både Huffman-kodning och LZW-kodning är bra metoder för komprimering av data. De båda har sina fördelar respektive nackdelar. Nästan i alla fall kan vi komprimera filen mer om vi använder LZW-kodning istället för Huffman-kodning. Det beror på att med Huffman-kodning måste vi sparar frekvenstabellen, vilket tar extra plats i den kodade filen. Den främsta anledningen är dock att i min implementation av Huffman-kodningen kodar vi bara en symbol i taget och tar inte hänsyn till källans minne av symboler, vilket vi gör i LZW-kodning.

I varje fall med Huffman-kodning kommer vi väldigt nära entropin (optimala takten) som kan presteras då vi kodar med en symbol i taget. Så Huffman-koden presterar relativt sätt väldigt bra. I LZW-koden är takten ungefär lika med den betingade entropin då vi vet en symbol tillbaka i tiden. Detta betyder alltså att vi ungefär kodar med dubbla symboler i LZW-koden.

Utifrån detta experiment är det svårt att dra slutsats om vilka filtyper som passar bättre till de olika kodningsmetoderna. Huffman-kodningen passar bra till de flesta typer av filer eftersom takten är nära entropin i alla fall av filtyper. LZW-kodning drar fördelar av återupprepning av symboler, vilket betyder att den presterar bäst på filer med upprepade symboler. Det kan vara både txt-, gif- eller pngfiler.

%%%%%%%%%%%%%%%%%%%%%%%%%%%%%%%%%%%%%%%%%%%%%%%%%%%%%%%%%%%%%%%%%%%%%%
%%% lorem.tex ends here

%%% Local Variables: 
%%% mode: latex
%%% TeX-master: "demothesis"
%%% End: 
